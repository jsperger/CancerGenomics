\documentclass{article}
\author{
Courtney Vaughn, Isaac Robson, Kentaro Hoffman, and John Sperger
%  Hoffman, Kentaro \and
%  Sperger, John  \and
%  Robson, Isaac  \and
%  Vaughn, Courtney 
}
\title{Cancer Genomics - Analysis Plan}
\date{March $19^{\mathrm{th}}$, 2018}
\usepackage{booktabs}
\usepackage{float}
\begin{document}
\maketitle

\section{Introduction}

   Our group will analyze the publically available serous ovarian cancer data available through The Cancer Genome Atlas. The TCGA data documents patient response to Platinum-based chemotherapies in terms of survival time with treatment and progression-free survival. All patients receive very similar treatments so a longer survival time implies that the patients responded better to the drug. We aim to investigate two potential pathways for improved drug response: How does expression of DNA damage response factors influence survival? Does the immune system play a role in response to platinums, and if so, which immune pathways improve/reduce survival? 
  
   Past work by The Cancer Genome Atlas Research Network ``delineated four ovarian cancer transcriptional subtypes, three microRNA subtypes, four promoter methylation subtypes and a transcriptional signature associated with survival duration'' \cite{cancer2011integrated}. We will first reproduce the transcriptional subtype cluster analysis to ensure there are no issues with the data. We will then perform two additional cluster analyses based on DNA damage response factors and immune inhibitors. We will then use the clinical data, existing subtype classification, and the classifications from our additional clusters to predict progression-free survival and overall survival using the Cox model. To reduce dimensionality issues we will also perform variable selection using elastic net which combines the $L_1$ and $L_2$ penalties from LASSO and Ridge Regression \cite{zou2005regularization}.  

\section{Data \& Pre-processing}

\begin{table}[h!] \centering
  \begin{tabular}{ll} \toprule
  Data Type & Cases \\
  \hline
  Clinical & 587 \\
  DNA Methylation & 602 \\
  Copy Number Variation & 573 \\
  RNA-Seq & 376 \\
  \bottomrule
  \end{tabular}
\caption{TCGA Ovarian Data Summary}
\end{table}

The above table summarizes the available data for the TCGA-OV project. We will restrict the sample to only those with clinical data available. 

\section{Analysis Plan}

\subsection{Subtype Replication}
   Our first planned analysis is a reproduction of the hierarchical clustering analysis\cite{eisen_cluster_1998} done on the mRNA data to replicate the four substypes identified in the initial report by the TCGA network \cite{cancer2011integrated}. This will show that the intial results are reproducible, and help ensure that any additional results we may discover are not due to differences in the pre-processing stage. 
   
%Euclidean distance
\subsection{Identify gene list of DNA damage response genes}
(I know where to get these and we'll have plenty genes to do clustering with)

\subsection{Clustering by DNA Damage Response}
  Based on the gene list we identify, we will make clusters based on RNA levels of genes involved in DNA damage response. We will first create a hierarchical clustering as an exploratory analysis, and then will use consensus clustering to select the final number of clusters \cite{monti_consensus_2003}. It has been hypothesisized (cite?)there there is a role for repair/ other damage response pathways in drug effectiveness. 

\subsection{Mining differential correlation} 
Differential Corrrelation Mining (DCM) is a method developed by Kelly Bodwin, Kai Zhang, and Andrew Nobel for identifying sets of variables where the average pairwise correlation between variables in a set is higher under one sample condition than the other \cite{difcor2016} . We will split the mRNA-seq data into clusters based on the subgroups identified in the previous stage of the analysis and run DCM to identify sets of differentially correlated genes across subtypes. We will then investigate the gene sets to identify potential underlying biological mechanisms. Existing analyses focus on gene counts, and it is possible for counts to remain similar while correlations change. This is potentially interesting because differential correlation could indicate that a group of genes which normally work together aren't functioning together in the same way in a particular subtypes. 


\subsection{Survival Analysis}
We will fit Kaplan-Meier Survival curves for each of the DNA damage response subtypes we identify to estimate the unadjusted 

Next we will fit a Cox proportional hazards model with the DNA damage response subtypes and clinical variables as predictorsto estimate the effect of the DNA damage response subtypes on survival controlling for clinical factors like age. To better understand the predictive accuracy and the robustness of the model, we will split the data into training and test data sets using a 70/30 train/test split and report the prediction error on the test set. 

\subsection{Repair Pathway}

If there are differences in survival between subgroups we will follow up to see which repair pathways are most important to drug response. This is an interesting question since people often assume repair is a mechanism of resistance to platinums but there really isn't any strong evidence for this.

\subsection{Clustering by Immune Inhibitors}

We know that there are synergistic effects between platinums and immune inhibitors but the relationship between these has not been fleshed out fully.  We can cluster samples based on immune response related genes and then see if there is a difference in survival based on these. The follow-up we can look at which pathways may be associated with decreased response to platinums and see if there are drugs that modulate those. The original TCGA paper did define a subtype of ovarian cancer as "immune responsive" and we can compare if our clustering just sorts according to that subtype or if it suggests additional sub-subtypes. 

\bibliography{genomics}{}
\bibliographystyle{plain}
\end{document}
